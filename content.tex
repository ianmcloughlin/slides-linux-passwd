\begin{frame}{passwd}
  
  \begin{center}
    \mintinline{bash}{joeb:x:1000:1000:Joe Bloggs:/home/joeb:/bin/bash}
  \end{center}
  \vspace{3mm}
  
  \begin{description}
    \setlength\itemsep{2mm}
    \item[/etc/passwd] stores information about users for logins.
    \item[Passwords] are not stored in it.
    \item[Historically] passwords were stored in the \mintinline{python}{x} part.
    \item[All users] have read (but not write) access to passwd.
    \item[Shadow] passwords are now used instead.
  \end{description}

\end{frame}


\begin{frame}{shadow}
  
  \begin{center}
    \mintinline{bash}{joeb:salt+hashedpasswordhere:17895:0:99999::::}
  \end{center}
  \vspace{3mm}
  
  \begin{description}
    \setlength\itemsep{6mm}
    \item[/etc/shadow] stores users hashed passwords.
    \item[Normal users] cannot read the file, only root users.
    \item[Not perfect] but another layer of security.
    \item[Why] not just do this to passwd?
  \end{description}

\end{frame}


\begin{frame}{Hashing passwords}
  
  \begin{center}
    \mintinline{bash}{$1$ie$sVzlF54.Tz0JuVRDJ3PiK.}
  \end{center}
  \vspace{3mm}

  \begin{description}[xxxx]
    \setlength\itemsep{6mm}
    \item[1] means MD5 was used. We usually now use SHA512, and 1 becomes 6.
    \item[ie] is the salt. It makes dictionary cracking more difficult.
    \item[Rest] is hashed password, but with some techniques like key stretching added.
  \end{description}

\end{frame}
